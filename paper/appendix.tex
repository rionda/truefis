\appendix
\section{Appendix}
We now present the proof of Thm.~\ref{thm:eapproxempir}. Before we start, we
need to introduce some additional settings and terminology.

As in Sect.~\ref{sec:prelvcdim}, let $D$ be a domain and $\range$ be a
collection of subsets from $D$. Let $S=\{t_1,\dotsc,t_\ell\}$ be a bag of $\ell$
elements from $D$. For each $t_i$, $1\le i\le \ell$, let $\sigma_i$ be a
Rademacher random variable, i.e., a random variable taking value $-1$ or $1$,
each with probability $1/2$. The random variables $\sigma_i$ are independent.
For any $r\in\range$, the quantity
\[
	R_S(r)=\frac{1}{\ell}\sum_{i=1}^\ell \sigma_i\mathds{1}_r(t_i)
\]
is known as the \emph{sample Rademacher average} of $r$ in the set $S$. The
\emph{conditional Rademacher average} is the quantity
\[
	\mathsf{R}_S=\mathbb{E}_\sigma\left[\sup_{r\in\range}R_S(r)\right],
\]
where $\mathbb{E}_\sigma$ denotes the expectation taken only w.r.t.~the random
variables $\sigma_i$, $1\le i\le\ell$ (i.e., conditionally on the sample).

Massart's lemma gives a bound to $\mathsf{R}_S$ and is usually presented as follows.
\begin{lemma}[Massart's lemma, Thm.~3.3~\citep{BoucheronBL05}]\label{lem:massart}
	Let $S$ be a finite subset of $D$ of size $|S|=\ell$. Then
	\[
		\mathsf{R}_S\le\max_{r\in\range}|r\cap S|\frac{\sqrt{2\ln
		|P_\range(S)|}}{\ell},
	\]
	where $P_\range(S)$ is the projection of $\range$ on $S$, i.e., the set
	$\{S\cap r ~:~ r\in\range\}$.
\end{lemma}

The Sauer-Shelah lemma, a celebrated result in combinatorics, gives an upper
bound to the size of the projection $P_\range(S)$, provided an upper bound to
the (empirical) VC-dimension is known. It allows us to use bounds to the
(empirical) VC-dimension in Massart's lemma.

\begin{lemma}[Sauer-Shelah Lemma, Lemma 6.10~\citep{ShalevSBD14}]\label{lem:sauer}
	Let $\range$ be a range set on a domain $D$, and let $\VC(\range)\le d$.
	Then, for any finite $S\subseteq D$ of size $|S|=\ell$, we have that
	\[
		|P_\range(S)|\le \sum_{i=0}^d\binom{\ell}{i}\enspace.
	\]
	In particular, if $\ell\ge d+1$, then $|P_\range(S)|\le (e\ell/d)^d$.

	The result also holds if the upper bound $d$ to $\VC(\range)$ is replaced with
	an upper bound $d_S$ to $\EVC(\range,S)$.
\end{lemma}

Theorem.~\ref{thm:eapproxempir} follows from Massart's Lemma, the Sauer-Shelah
lemma and the following result, which is at the core of statistical learning
theory.

\begin{theorem}[Thm.~3.3~\citep{BoucheronBL05}]
	Let $S$ be a bag of $\ell$ elements from $D$, sampled independently
	according to the distribution $\nu$.  Then, with probability at least
	$1-\delta$,
	\[
		\sup_{r\in\range}|\nu(r)-\nu_S(r)|\le 2 \mathsf{R}_S
		+\sqrt{\frac{2\ln\frac{2}{\delta}}{\ell}}\enspace.
	\]
\end{theorem}
