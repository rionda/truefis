\begin{abstract} 
   Frequent Itemsets (FIs) mining is a fundamental primitive in knowledge
   discovery. It requires to identify all itemsets appearing in at least a
   fraction $\theta$ of a transactional dataset $\Ds$. Often though, the
   ultimate goal of mining $\Ds$ is not an analysis of the dataset \emph{per
   se}, but the understanding of the underlying process that generated it.
   Specifically, in many applications $\Ds$ is a collection of samples obtained
   from an unknown probability distribution $\prob$ on transactions, and by
   extracting the FIs in $\Ds$ one attempts to infer itemsets that are
   frequently (i.e., with probability at least $\theta$) generated by $\prob$,
   which we call the True Frequent Itemsets (TFIs). Due to the inherently
   stochastic nature of the generative process, the set of FIs is only a rough
   approximation of the set of TFIs, as it often contains a huge number of
   \emph{false positives}, i.e., spurious itemsets that are not among the TFIs.
   In this work we design and analyze an algorithm to identify a threshold
   $\hat{\theta}$ such that the collection of itemsets with frequency at least
   $\hat{\theta}$ in $\Ds$ contains only TFIs with probability at least
   $1-\delta$, for some user-specified $\delta$. Our method uses results from
   statistical learning theory involving the (empirical) VC-dimension of the
   problem at hand. This allows us to identify almost all the TFIs without
   including any false positive.  We also experimentally compare our method with
   the direct mining of $\Ds$ at frequency $\theta$ and with techniques based on
   widely-used standard bounds (i.e., the Chernoff bounds) of the binomial
   distribution, and show that our algorithm outperforms these methods and
   achieves even better results than what is guaranteed by the theoretical
   analysis.
 \end{abstract}

%{\bf Categories and Subject Descriptors:} H.2.8 [Database Management]: Database Applications -- \emph{Data Mining}

{\bf Keywords:} Frequent itemsets, VC-dimension, False positives,
Distribution-free methods, Frequency threshold identification, Pattern mining,
Significant patterns.

