The intuition behind the method is the following. Let $\mathcal{B}$ be the
\emph{negative border} of $\TFI(\prob,\Itm,\theta)$, that is the set of itemsets
not in $\TFI(\prob,\Itm,\theta)$ but such that all their proper subsets are in
$\TFI(\prob,\Itm,\theta)$. If we can find an $\varepsilon$ such that $\Ds$ is an
$\varepsilon$-approximation to $(\range(\mathcal{B}),\pi)$ then we have that any
itemset $A\in\mathcal{B}$ has a frequency $f_\Ds(A)$ in $\Ds$ less than
$\hat{\theta}=\theta+\varepsilon$, given that it must be $\tfreq(A)<\theta$. By
the antimonotonicity property of the frequency, the same holds for all itemsets
that are supersets of those in $\mathcal{B}$. Hence, the only itemsets that can
have frequency in $\Ds$ greater or equal to $\hat{\theta}=\theta+\varepsilon$
are those with true frequency at least $\theta$. In the following paragraphs we
show how to compute $\varepsilon$.

\paragraph{Computing a superset of $\mathcal{B}$.} The first step of our
algorithm, which we call \ALG{}, is to compute a superset of $\mathcal{B}$, the
negative border of $\TFI(\prob,\Itm,\theta)$. Let $\delta_1$ and $\delta_2$ be
such that $(1-\delta_1)(1-\delta_2)\ge 1-\delta$. Let $\range(2^\Itm)$ be the
range space of all itemsets. \fabio{Let $d$ be the upper bound to the empirical
VC-dimension of $\range(2^\Itm)$ on $\Ds$ (the computation of $d$ is discussed in
Sect.~\ref{sec:computvc}).} We can then use~\eqref{eq:evceapproxitemsets}
with $\delta_1$, the bound $d$, and the maximum frequency of an item in $\Ds$ to
compute an $\varepsilon_1$ such that $\Ds$ is, with probability at least
$1-\delta_1$, an $\varepsilon_1$-approximation to $(\range(2^\Itm),\prob)$. Let
now $\mathcal{W}$ be the \emph{negative border} of
$\mathcal{C}_1=\FI(\Ds,\Itm,\theta-\varepsilon_1)$, and let
$\mathcal{G}=\{A\subseteq\Itm ~:~ \theta-\varepsilon_1\le
f_\Ds(A)<\theta+\varepsilon_1\}$, and $\mathcal{F}=\mathcal{G}\cup\mathcal{W}$.

\begin{fact}\label{fact:supersetborder}
	If $\Ds$ is a $\varepsilon_1$-approximation to $(\range(2^\Itm),\prob)$,
	then $\mathcal{B}\subseteq\mathcal{F}$.
\end{fact}

Now that we
have a superset $\mathcal{F}$ of $\mathcal{B}$, we want to compute an upper
bound to the empirical VC-dimension of $\mathcal{B}$ on $\Ds$. 
\fabio{ This can be done by computing the empirical VC-dimension of $\mathcal{F}$ on $\Ds$,
or an upper bound to it.
Algorithms computing upper bounds for $\EVC(\range(\mathcal{C}),\Ds)$
 where $C$ is a collection of itemsets, are presented in Section~\ref{sec:computvc}.
However, a better bound can be obtained by taking into account the structure
of $\mathcal{B}$. Algorithms that compute upper bounds for $\EVC(\range(\mathcal{B}),\Ds)$ (i.e., specific for 
the negative border $\mathcal{B}$ of $\TFI(\prob,\Itm,\theta)$) are presented in
Section~\ref{sec:vcbupper} In the remaining of this section, we only
assume that we have computed an upper bound $d_2$ to $\EVC(\range(\mathcal{B}),\Ds)$.
We can use $d_2$ in~\eqref{eq:evceapproxitemsets}
together with $\delta_2$ and the maximum frequency in $\Ds$ of an itemset of
$\mathcal{F}$, to obtain an $\varepsilon_2$ such that, with probability
at least $1-\delta_2$, $\Ds$ is an $\varepsilon_2$-approximation to
$(\range(\mathcal{B}),\prob)$. Let $\hat{\theta}=\theta+\varepsilon_2$. The
following Theorem shows that $\hat{\theta}$ has the desired properties.
Algorithm~\ref{alg:vcfull} presents the pseudocode of our algorithm.
}

\begin{theorem}\label{lem:vcfull}
With probability at least $1-\delta$, $\FI(\Ds,\Itm,\hat\theta)$ contains no
false positives:
\[
\Pr\left(\FI(\Ds,\Itm,\hat\theta)\subseteq\TFI(\prob,\Itm,\theta)\right)\ge 1-\delta\enspace.\]
\end{theorem}
\begin{proof}
  Consider the two events $\mathsf{E}_1$=``$\Ds$ is an
  $\varepsilon_1$-approximation for $(\range(2^\Itm),\prob)$'' and
  $\mathsf{E}_2$=``$\Ds$ is an $\varepsilon_2$-approximation for
  $(\range(\mathcal{B}),\prob)$''. From the above discussion and the definition
  of $\delta_1$ and $\delta_2$ it follows that the event
  $\mathsf{E}=\mathsf{E}_1\cap\mathsf{E}_1$ occurs with probability at least
  $1-\delta$. Suppose from now on that indeed $\mathsf{E}$ occurs.

  Since $\mathsf{E}_1$ occurs, then Fact~\ref{fact:supersetborder} holds, and
  the value $d_2$ is effectively an upper bound to $\EVC(\range(\mathcal{B},\Ds)$

  Since $\mathsf{E}_2$ also occurs, then for any $A\in\mathcal{B}$ we have
  $|\tfreq(A)-f_\Ds(A)|\le\varepsilon_2$, but given that $\tfreq(A)<\theta$
  because the elements of $\mathcal{B}$ are not TFIs, then we have
  $f_\Ds(A)<\theta+\varepsilon_2$. Because of the antimonotonicity property of
  the frequency and the definition of $\mathcal{B}$, this holds for any itemset
  that is not in $\TFI(\prob,\Itm,\theta)$. Hence, the only itemsets that can
  have a frequency in $\Ds$ at least $\hat{\theta}=\theta+\varepsilon_2$ are the
  TFIs, so $\FI(\Ds,\Itm,\hat{\theta})\subseteq\TFI(\prob,\Itm,\theta)$, which
  concludes our proof.
\end{proof}

\begin{algorithm}[htbp]
  \SetKwInOut{Input}{Input}
  \SetKwInOut{Output}{Output}
  \SetKwComment{Comment}{\quad// }{}
  \SetKwFunction{SolveAntichainSUKP}{solveAntichainSUKP}
   \DontPrintSemicolon
   \Input{Dataset $\Ds$, freq.~threshold $\theta\in(0,1)$, confidence
   $\delta\in(0,1)$}
  \Output{Freq.~threshold $\hat{\theta}$
  s.~t.~$\FI(\Ds,\Itm,\hat{\theta})$ contains only TFIs with prob.~at least
  $1-\delta$.}
  $\delta_1,\delta_2\leftarrow 1-\sqrt{1-\delta}$ \Comment{$\delta_1$ and $\delta_2$ do not need to have the same value}
  $d_1\leftarrow$ upper bound to $\EVC(\range(2^\Itm),\Ds)$  \Comment{as
	discussed in Sect.~\ref{sec:computvcexact}}
  $\varepsilon_1\leftarrow 2c\sqrt{\frac{2d_1\max_{a\in\Itm}f_\Ds(\{a\})}{|\Ds|}}
  + \sqrt{\frac{2\ln\frac{4}{\delta_1}}{|\Ds|}}$\;
  $\mathcal{C}_1=\FI(\Ds,\Itm,\theta-\varepsilon_1)$\;
  $\mathcal{G}=\{A\subseteq\Itm ~:~ \theta-\varepsilon_1\le
f_\Ds(A)<\theta+\varepsilon_1\}$\;
  $\mathcal{W}\leftarrow$ negative border of $\mathcal{C}_1$\;
  $\mathcal{F}=\mathcal{G}\cup\mathcal{W}$\;
  $d_2\leftarrow$ upper bound to $\EVC(\range(\mathcal{B}),\Ds)$ \Comment{using
  an ``antichain-aware'' algorithm on $\mathcal{F}$}
  $\varepsilon_2\leftarrow 2c\sqrt{\frac{2d_2\max_{A\in\mathcal{F}}f_\Ds(A)}{|\Ds|}}
  + \sqrt{\frac{2\ln\frac{4}{\delta_2}}{|\Ds|}}$\;
  \Return{$\theta+\varepsilon_2$}
  \caption{Compute freq.~threshold $\hat{\theta}$
  s.~t.~$\FI(\Ds,\Itm,\hat{\theta})$ contains only TFIs with prob.~at least
  $1-\delta$.}
  \label{alg:vcfull}
\end{algorithm}

% THE FOLLOWING NO LONGER APPLIES BECAUSE WE DON'T COMPUTE THE NON-EMP VC.
%\paragraph{Exploiting additional knowledge about $\prob$.} Our algorithm is
%completely \emph{distribution-free}, i.e., it does not require any assumption
%about the unknown distribution $\prob$. On the other hand, when information
%about $\prob$ is available, our method can exploit it to achieve better
%performances in terms of running time, practicality, and accuracy.  For example,
%in most applications $\prob$ will not generate any transaction longer than some
%known upper bound $\ell\ll|\Itm|$. Consider for example an online marketplace
%like Amazon: it is extremely unlikely (if not humanly impossible) that a single
%customer buys one of each available product. Indeed, given the hundred of
%thousands of items on sale, it is safe to assume that all the transactions will
%contains at most $\ell$ items, for some $\ell\ll|\Itm|$. Other times, like in an
%online survey, it is the nature of the process that limits the number of items
%in a transaction, in this case the number of questions. A different kind of
%information about the generative process may consists in knowing that some
%combination of items may never occur, because ``forbidden'' in some wide sense.
%Other examples are possible. All these pieces of information can be used to
%compute better (i.e., stricter) upper bounds to the VC-dimension
%$\VC(\range(2^\Itm))$. For example, if we know that $\prob$ will never generate
%transactions with more than $\ell$ items, we can safely say that
%$\VC(\range(2^\Itm))\le \ell$, a much stricter bound than $|\Itm|-1$ from
%Corol.~\ref{thm:vcdimubfirst}. This may result in a smaller $\varepsilon_1$, a
%smaller $\varepsilon$, and a smaller $\hat\theta$, which allows to produce more
%TFIs in the output collection. In the experimental evaluation, we show the
%positive impact of including additional information may on the performances of
%our algorithm.

\subsection{\fabio{Computing upper bounds to the empirical VC-Dimension}}\label{sec:computvc}
\fabio{We now present different methods to compute upper bounds to the empirical
VC-dimension $\EVC(\range(\mathcal{C},\Ds)$ of an
itemset collection $\mathcal{C}$ in a dataset $\Ds$. Our algorithms are based
on the to the computation of the d-index of $\mathcal{C}$ on $\Ds$.
We start from an exact algorithm which may be too expensive in practice
and then discuss a number of variants that trade off accuracy for speed.}

\subsubsection{Exact algorithm and faster variants}\label{sec:computvcexact}
The exact algorithm (which we call \texttt{EVCBoundExact}) \fabio{to compute} the d-index of
$\mathcal{C}$ on $\Ds$ starts by scanning $\Ds$ and computing, for each
transaction $\tau\in\Ds$, the set $I_\tau=\tau\cap U$. If $I_\tau\neq U$ and
$I_\tau\neq\emptyset$, then we add $I_\tau$ to a set $L$. Let now
$\mathcal{C}_\tau$ be the sets of itemsets of $\mathcal{C}$ that appear in
$\tau$. Now, for any $d\in\mathbb{N}$ let
\[
	T_d=\{I_\tau\in L ~:~ \tau\in\Ds \wedge \lfloor\log_2|\mathcal{C}_\tau|\rfloor+1= d\},\mbox{ and } \mathsf{T}_d=\bigcup_{i\ge d} T_d
\]
Once the scan is complete, let $d_1$  be the maximum integer for which
$|\mathsf{T}_d|\ge d$, i.e., $\fabio{d_1}=\max\{d ~:~ |\mathsf{T}_d|\ge d\}$. Let now
$b_1$ be the size of the largest antichain on $\mathsf{T}_{d_1}$, obtained by
solving the corresponding maximum bipartite matching problem. If $b_1\ge d_1$,
then the algorithm can return \fabio{$d_1$}. Otherwise, for $i>1$, let $d_i$ be the
maximum $d<d_{i-1}$ such that \fabio{$|\mathsf{T}_d\setminus\mathsf{T}_{d_{i-1}}|\ge d-b_{i-1}$}, where $b_{i-1}$ is the size
of the largest antichain on $\mathsf{T}_{d_{i-1}}$:
\[
	d_i=\max\{d<d_{i-1} ~:~ |\mathsf{T}_d\setminus\mathsf{T}_{d_{i-1}}|\ge d-b_{i-1}\}\enspace.
\]
We then compute the size $b_i$ of the largest antichain on $\mathsf{T}_{d_i}$
and return $b_i$ if $b_i\ge d_i$. Note that if $d_i\le b_{i-1}$, we can actually
avoid computing $b_i$ and return $d_i$ immediately. \fabio{The intuition behind this is
that since $\mathsf{T}_{d_{i-1}} \subset \mathsf{T}_{d_{i}}$, we already know that $\mathsf{T}_{d_{i-1}}$
contains an antichain of cardinality $b_{i-1}$.
[Matteo: please check, and add a couple of sentences 
to give the idea of what is going on, if needed.]}

The pseudocode of \texttt{EVCBoundExact} is presented in
Alg.~\ref{alg:evcboundexact}. The algorithm obviously terminates because in the
worst case there will be an index $i$ such that $d_i=1$, and any set containing
a single transaction is an antichain of size 1, so $b_i=d_1=1$, and the
algorithm terminates.

\begin{algorithm}[htbp]
  \SetKwInOut{Input}{Input}
  \SetKwInOut{Output}{Output}
  \SetKwComment{Comment}{\quad// }{}
  \SetKw{MyAnd}{and}
  \SetKw{MyTo}{to}
  \DontPrintSemicolon
  \Input{Dataset $\Ds$, collection $\mathcal{C}$ of itemsets.}
  \Output{d-index of $\mathcal{C}$ in $\Ds$.}
  $U\leftarrow\{a\in A, A\in\mathcal{C}\}$\;
  $L\leftarrow\emptyset$\;
  $T\leftarrow$ empty hash table\;
  \For{$\tau\in\Ds$}{\label{algline:forbegin}
		$I_\tau\leftarrow \tau\cap U$\;
		\If{$I_\tau\neq U$ \MyAnd $I_\tau\neq\emptyset$ \MyAnd $I_\tau\not\in L$} {
			$\mathcal{C}_\tau\leftarrow \emptyset$\;
			\For{$A\in\mathcal{C}$}{\label{algline:intersectionloopbegin}
				\If{$A\subseteq I_\tau$}{
					$\mathcal{C}_\tau\leftarrow \mathcal{C}_\tau\cup\{A\}$\;
				}
			}\label{algline:intersectionloopend}
			$L\leftarrow L\cup\{I_\tau\}$\;
			$\ell_\tau=|\mathcal{C}_\tau|$\;\label{algline:assignmentintersection}
			\eIf{$T$ does not contain any element with key
				$\lfloor\log_2\ell_\tau\rfloor + 1$}{
					$T[\lfloor\log_2\ell_\tau\rfloor + 1]\leftarrow \{I_\tau\}$\;
			}{
				$T[\lfloor\log_2\ell_\tau\rfloor + 1]\leftarrow
				T[\lfloor\log_2\ell_\tau\rfloor + 1]\cup\{I_\tau\}$\;
			}
		}
  }\label{algline:forend}
  $b_0\leftarrow 0$\;\label{algline:beginreplace}
  $d_0\leftarrow +\infty$\;
  $T[d_0]\leftarrow\emptyset$\;
  $i\leftarrow 1$\;
  \While{True} {\label{algline:secondloopbegin}
  \fabio{$d_i=\max\{ d < d_{i-1} ~:~ T[d] - T[ d-1]\ge d-b_{i-1}\}$}\;
	\If{$d_i\le b_{i-1}$}{
		\Return{$d_i$}\;
	}
	\fabio{$b_i\leftarrow$ size of largest antichain on
	$T[d_i]$} \;\label{algline:antichain}
	\If{$b_i \ge d_i$}{
		\Return{$d_i$}
	}
	\fabio{$i \leftarrow i+1$}\;
  }\label{algline:secondloopend}
  \caption{\texttt{EVCBoundExact}: compute the d-index of $\mathcal{C}$ in
  $\Ds$.}
  \label{alg:evcboundexact}
\end{algorithm}

\begin{lemma}\label{lem:evcboundexactcorrect}
	\texttt{EVCBoundExact} computes the d-index of $\mathcal{C}$ in $\Ds$.
\end{lemma}
\begin{proof}
	The algorithms stops in two cases: either there is an index $i$ such that
	$b_i\ge d_i$, or there is an index $i$ such that $d_i\le b_{i-1}$. We now
	consider the two cases separately.

	\begin{description}
		\item{Case 1.} Let $i$ be the index such that $b_i\ge d_i$. Then, there
			is an antichain of size at least $d_i$ on $\mathsf{T}_{d_i}$.
			$\mathsf{T}_{d_i}$ contains all and only the sets of the form
			$I_\tau=\tau\cap U$, where $\tau$ is a transaction in $\Ds$, such
			that $I_\tau$ (and therefore $\tau$) contains \emph{at least}
			$2^{d_i-1}$ itemsets from $\mathcal{C}$. From the definition of the
			sequence $d_j$, we have that $d_i$ is the maximum value for which we
			can find a collection of $d_i$ transactions
			$\{\tau_1,\dotsc,\tau_{d_i}\}\subseteq\Ds$ for which the sets
			$\tau_j\cap U$ form an antichain and such that each of these sets
			contains at least $2^{d_i-1}$ itemsets of $\mathcal{C}$. This is
			exactly the definition of the d-index of $\mathcal{C}$ on $\Ds$,
			hence the algorithm returns the correct value in this case.
		\item{Case 2.} Let $i$ be the index such that $d_i\le b_{i-1}$. From the
			definition of $b_{i-1}$, we know that there is an antichain of size
			$b_{i-1}$ on $\mathsf{T}_{d_{i-1}}$. Since
			$\mathsf{T}_{d_{i-1}}\subseteq \mathsf{T}_{d_i}$, there is an
			antichain of size $b_{i-1}$ also on $\mathsf{T}_{d_1}$ (the same
			antichain). So there is an antichain of size at least $d_i$ on
			$\mathsf{T}_{d_i}$. The proof then follows as in the previous case.
	\end{description}
\end{proof}

\paragraph{Trade-off accuracy for speed.} The complexity of
\texttt{EVCBoundExact} is clearly polynomial in the sizes of $\Ds$,
$\mathcal{C}$, and $U$, but indeed some steps are quite expensive. \fabio{In
particular, the following two steps can be modified to obtain a faster algorithm, at the expenses of
obtaining an \emph{upper bound to the d-index} instead of the exact value:}
\begin{enumerate}
	\item each iteration of the loop on
		lines~\ref{algline:secondloopbegin}--\ref{algline:secondloopend} in
		Alg.~\ref{alg:evcboundexact} requires to solve a maximum bipartite
		matching problem which can be done in polynomial time but may still be
		costly in practice;
	\item the loop on
		lines~\ref{algline:intersectionloopbegin}--\ref{algline:intersectionloopend}
		requires to iterate over each element of $\mathcal{C}$, which can be
		very large.
\end{enumerate}

To solve the first issue we can modify the algorithm to simply return $d_1$,
without having to solve any maximum bipartite matching problem. We call the
resulting algorithm \texttt{EVCBoundScan}. It is easy to see that the returned
value is still an upper bound to the empirical VC-dimension of $\mathcal{C}$ on
$\Ds$, as it is the maximum $d$ for which there are at least $d$ transactions
each containing at least $2^{d-1}$ itemsets of $\mathcal{C}$. In other words, by
returning $d_1$ we ignore the antichain constraint. The running time is greatly
improved. Any other value $d_i$ for which $b_i < d_i$ can also be returned,
giving the user some flexibility between the number of maximum bipartite
matching to solve and the optimality of the bound.
%We can replace lines~\ref{algline:beginreplace}--\ref{algline:secondloopend}
%with the pseudocode in Algorithm~\ref{alg:replacement}.
%
%\begin{algorithm}[hbt]
%	\DontPrintSemicolon
%	$d\leftarrow 1$\;
%	$\kappa\leftarrow$ any key in $L$\;
%	$\mathcal{T}\leftarrow\{\kappa\}$\;
%	\ForEach{key $k$ in $L$, except $\kappa$}{
%		\If{$L[k]\ge 2^{d-1}$}{
%			$\mathcal{R}\leftarrow\mathcal{T}\cup\{k\}$\;
%			$d\leftarrow$ maximum integer $q$ such that $\mathcal{R}$ contains
%			at least $q$ elements $c$ with $L[c]\ge 2^{q-1}$\;
%			$\mathcal{T}\leftarrow$ set of the $d$ elements of $\mathcal{R}$ with
%			largest values in $L$ (ties broken arbitrarily)\;
%		}
%	}
%	\Return{$d$}\;
%	\caption{Replacement for
%		lines~\ref{algline:beginreplace}--\ref{algline:secondloopend} in
%		Alg.~\ref{alg:evcboundexact}.}
%	\label{alg:replacement}
%\end{algorithm}
%
%The correctness of the algorithm with this replacement (we denote the resulting
%algorithm as \texttt{EVCBoundScan}) follows from the fact that at any iteration
%of the loop, the set $\mathcal{T}$ always contains $d$ elements of the form
%$I_\tau=\tau\cap U$ for $\tau\in\Ds$ such that each element in $\mathcal{T}$
%contains at least $2^{d-1}$ itemsets from $\mathcal{C}$ and $d$ is the maximum
%integer for which such a set exists, among the keys of $L$ seen up to the
%current iteration. This means that the value returned by \texttt{EVCBoundScan}
%is an upper bound to the d-index, as we are relaxing the antichain requirement.

%The running time is greatly improved compared to \texttt{EVCBoundScan} because
%now this part of the algorithm requires a single scan of the keys in $L$, and
%the work per key is minimal.

As for the loop on
lines~\ref{algline:intersectionloopbegin}--\ref{algline:intersectionloopend} of
Alg.~\ref{alg:evcboundexact}, we can use $2^{|I_\tau|}$ as an upper bound to the
exact number of itemsets of $\mathcal{C}$ that appear in $\mathcal{C}$, and
so-modified algorithm will clearly return an upper bound to the d-index. The
algorithm with both modification, which we call \texttt{EVCBoundScanFast},
requires time approximately proportional to two scans of the dataset, at the
price of a obtaining a looser bound to the empirical VC-dimension of
$\mathcal{C}$ on $\Ds$.

%\paragraph{The case of all itemsets.} The case $\mathcal{C}=2^{\Itm}$ is of
%special interest for us. In this case, $U=\Itm$, and each transaction $\tau$
%contains exactly $2^{|\tau|}-1$ itemsets. Hence we can use
%\texttt{EVCBoundScanFast} or the almost equivalent algorithm
%by~\citet{RiondatoU14} to compute an upper bound to $\EVC(\range(2^\Itm),\Ds)$
%with a single linear scan of $\Ds$.

\subsubsection{Integer programming method}
We now present another algorithm to compute an upper bound to the d-index of an
itemset collection $\mathcal{C}$ on a dataset $\Ds$, and therefore to the
empirical VC-dimension of $\mathcal{C}$ on $\Ds$. \fabio{The intuition and the purpose
of this algorithm is to balance the trade-off between the accuracy of
\texttt{EVCBoundExact} and the computation time of \texttt{EVCBoundScanFast}.}
Indeed, \texttt{EVCBoundExact} uses all available information about
$\mathcal{C}$ to compute the exact set (and number) of itemsets from
$\mathcal{C}$ contained in each transaction. On the other hand,
\texttt{EVCBoundScanFast} almost completely ignores any ``structure'' of
$\mathcal{C}$, except for the set $U$ of items, and use the very loose upper bound
$2^{|\tau\cap U|}-1$ for the number of itemsets from $\mathcal{C}$ that appear
in a transaction $\tau$. The algorithm we present in this section tries to find
a stricter upper bound by exploiting the knowledge about $\mathcal{C}$, but
without having to compute the explicit set of itemsets of $\mathcal{C}$ in
$\tau$, and instead asking how many itemsets of $\mathcal{C}$ can appear in a
transaction $\tau$ for which $|\tau\cap U|=\ell$.

The algorithm, which we call \texttt{EVCBoundSUKP} requires to solve a \emph{Set
Union Knapsack Problem} (SUKP)~\citep{GoldschmidtNY94} associated to the
collection $\mathcal{C}$.

\begin{definition}[Set Union Knapsack Problem (SUKP)\citep{GoldschmidtNY94}]\label{def:sukp}
  Let $U=\{a_1,\dotsc,a_\ell\}$ be a set of elements and let
  $\mathcal{S}=\{A_1,\dotsc,A_k\}$ be a set of subsets of $U$, i.e.,
  $A_i\subseteq U$ for $1\le i\le k$. Each subset $A_i$, $1\le i\le k$, has an
  associated non-negative \emph{profit} $\rho(A_i)\in\mathbb{R}^+$, and each
  element $a_j$, $1\le j\le\ell$ as an associated non-negative weight
  $w(a_j)\in\mathbb{R}^+$.  Given a subset $\mathcal{S}'\subseteq\mathcal{S}$,
  we define the profit of $\mathcal{S}'$ as $P(\mathcal{S}')=\sum_{A_i\in
  \mathcal{S}'}\rho(A_i)$. Let $U_{\mathcal{S}'}=\cup_{A_i\in\mathcal{S}'}
  A_i$. We define the weight of $\mathcal{S}'$ as $W(\mathcal{S}')=\sum_{a_j\in
  U_{\mathcal{S}'}} w(a_j)$. Given a non-negative parameter $c$ that we call
  \emph{capacity}, the \emph{Set-Union Knapsack Problem} (SUKP) requires to find
  the set $\mathcal{S}^*\subseteq\mathcal{S}$ which \emph{maximizes}
  $P(\mathcal{S}')$ over all sets $\mathcal{S}'$ such that $W(\mathcal{S}')\le
  c$.
\end{definition}

In our case, $U$ is the set of items that appear in the itemsets of
$\mathcal{C}$, $\mathcal{S}=\mathcal{C}$, the profits and the weights are all
unitary, and the capacity constraint is an integer $\ell$. We call this
optimization problem the \emph{SUKP associated to $\mathcal{C}$ with capacity
$\ell$}. It is easy to see that the optimal profit $p$ of this SUKP is the
number of itemsets from $\mathcal{C}$ that can be built using $\ell$
items from $U$:
\[
	p=\max_{\substack{\mathcal{A}\subseteq\mathcal{C}\\ |\cup_{A\in\mathcal{A}}
A|\le\ell}} |\mathcal{A}|\enspace.
\]

Consider the set $\Ds_\mathcal{C} = \{ \tau \cap U, \tau\in \Ds\}$ (which is a
\emph{set}, not a bag, hence it does not contain duplicates, even if $\Ds$
contains multiple identical transactions). Let $\ell_1,\dotsc,\ell_w$ be the
sequence of the sizes of the elements of $\Ds_\mathcal{C}$, i.e., for each value
$\ell$ for which there is at least one element in $\Ds_\mathcal{C}$ of size
$\ell$, there is one (and only one) index $i$, $1\le i\le w$ such that
$\ell_i=\ell$. Assume that the $\ell_i$'s are labelled in sorted decreasing
order: $\ell_1>\ell_2>\dotsb>\ell_w$. Let now $L_i$, $1\le i\le w$ be the number
of elements of $\Ds_\mathcal{C}$ that have size at least $\ell_i$. %and such that
%for no two $\tau'$, $\tau''$ of them we have either $\tau'\subseteq\tau''$ or
%$\tau''\subseteq\tau'$.
Let now $q_i$ be the optimal profit of the SUKP associated to $\mathcal{C}$ with
capacity $L_i$, and let $b_i=\lfloor \log_2q_i\rfloor +1$. %The sequences
%$(\ell_i)_1^w$ and a sequence $(L_i^*)^w$ of upper bounds to $(L_i)_1^w$ can be
%computed efficiently with a scan of $\Ds_\mathcal{C}$.
The following lemma shows how to use these sequences to obtain an upper bound
to the empirical VC-dimension of $\mathcal{C}$ on $\Ds$. We call the resulting
method \texttt{EVCBoundSUKP}, and present its pseudocode in
Alg.~\ref{alg:evcboundsukp}.

\begin{lemma}\label{lem:sukpevc}
  Let $j$ be the minimum integer for which $b_i\le L_i$. Then
  $\EVC(\mathcal{C},\Ds)\le b_j$.
\end{lemma}
\begin{proof}
  If $b_j\le L_j$, then there are at least $b_j$ elements of $\Ds_\mathcal{C}$
  which can contain $2^{b_j-1}$ itemsets from $\mathcal{C}$ and this is the
  maximum $b_i$ for which it happens, because the sequence $b_1,b_2,\dotsc,b_w$
  is sorted in decreasing order, given that the sequence $q_1,q_2,\dotsc,q_w$
  is. Then $b_j$ satisfies the conditions of Lemma~\ref{lem:evcdimupbound}.
  Hence $\EVC(\mathcal{C},\Ds)\le b_j$.
\end{proof}

\begin{algorithm}[htbp]
  \SetKwInOut{Input}{Input}
  \SetKwInOut{Output}{Output}
  \SetKwFunction{SolveSUKP}{solveSUKP}
  \SetKwComment{Comment}{\quad// }{}
  \SetKw{MyAnd}{and}
  \DontPrintSemicolon
  \Input{Dataset $\Ds$, collection $\mathcal{C}$ of itemsets.}
  \Output{Upper bound to $\EVC(\range(\mathcal{C}),\Ds)$.}
  $U\leftarrow\{a\in A, A\in\mathcal{C}\}$\;
  $\Ds_\mathcal{C}\leftarrow\emptyset$\;
  $L\leftarrow$ empty hash table\;
  \For{$\tau\in\Ds$}{
		$I_\tau\leftarrow \tau\cap U$\;
		\If{$I_\tau\neq U$ \MyAnd $I_\tau\neq\emptyset$ \MyAnd $I_\tau
			\not\in\Ds_\mathcal{C}$} {
				$\Ds_\mathcal{C} \leftarrow \Ds_\mathcal{C} \cup \{I_\tau\}$\;
				\If{$L$ has no element with key $|I_\tau|$} {
					$L[|I_\tau|]\leftarrow 1$\;
				}
				\ForEach{key $k$ in $L$ such that $k<|I_\tau|$}{
					$L[k]\leftarrow L[k]+1$\;
			}
		}
  }
  $i\leftarrow 1$\;
  \While{$i\le$ number of keys in $L$}{
	$\ell_i\leftarrow $ $i$-th highest key in $L$\;
	$q_i\leftarrow$\SolveSUKP{$U$, $\mathcal{C}$, $\ell_i$}\;
	$b_i\leftarrow\lfloor\log_2 q_i\rfloor +1$\;
	\If{$b_i\le L[\ell_i]$}{
		\Return{$b_i$}\;
	}
	$i\leftarrow i+1$\;
  }
  \Return{$\min\{L[\ell_i],\log_2|\mathcal{C}|\}$}\;
  \caption{\texttt{EVCBoundSUKP}: compute an upper bound to
  $\EVC(\range(\mathcal{C}),\Ds)$.}
  \label{alg:evcboundsukp}
\end{algorithm}

\paragraph{Complexity and runtime considerations.} Solving the SUKP optimally
is NP-hard in the general case, although there are known restrictions for which
it can be solved in polynomial time using dynamic
programming~\citep{GoldschmidtNY94}. Since we have unit weights and unit
profits, our SUKP is equivalent to the \emph{densest $k$-subhypergraph} problem,
which can not be approximated within a factor of $2^{O(\log n)^\delta}$ for any
$\delta>0$ unless \fabio{$3-SAT \in
DTIME(2^{n^{3/4+\varepsilon}})$~\citep{HajiaghayiJKLMRSV06}.
[MATTEO: ref. 43 is not correct, the correct title (according to the LNCS
proceedings for IWBRA 2006) is: ``Minimum Multicolored Subgraph Problem in
Multiplex PCR Primer Set Selection and
Population Haplotyping''. Also, the paper in the proceedings does not contain the
result regarding 3SAT and DTIME. I have seen that in a draft of one of the authors 
webpage, but since it is not published I do not think we can refer to it.]}
 A greedy algorithm
by~\citet{Arulselvan14} allows a constant factor approximation if each item only
appears in a constant fraction of itemsets of $\mathcal{C}$.

Despite all these negative results, in our case it is actually \emph{not
necessary to compute the optimal solution} to the SUKP: any profit $p$ for which
we can prove that there is no power of two between $p$ and the optimal profit
would result in the same upper bound to the (empirical) VC-dimension, while
substantially speeding up the computation. This early-termination property can
be specified in currently available optimization problem solvers (e.g., CPLEX).
Moreover, \emph{any upper bound to the optimal profit can be used}, resulting in
a looser bound to the empirical VC-dimension. Optimization problem solvers
compute such bounds as part of their execution (using cutting-plane methods),
and can be instructed to terminate after a given amount of time and return the
best computed upper bound. Alternatively, it is possible to instruct
optimization problem solvers to return the first feasible solution whose profit
is proved to be within a certain gap from the optimal. As an example, if we
instruct CPLEX to return the first feasible solution whose profit $p$ is within
$50\%$ of the optimal profit $p^*$, then we can use $2p$ as an upper bound to
$p^*$. Since $p^*\le 2p\le 2p^*$ there is at most one power of two between $p^*$
and $2p$, so if we denote the value $\lfloor\log_2 2p\rfloor +1$ as $b$ and the
value $\lfloor\log_2 p^*\rfloor +1$ as $b^*$, we have that $b^*-b\le 1$, hence
the additional looseness in the bound to the empirical VC-dimension due to the
early termination of the optimization process is minimal.

These observations and the extreme scalability of modern optimization problem
solvers make it feasible to use \texttt{EVCBoundSUKP} even when $\mathcal{C}$
contains tens of thousands of itemsets.

One final comment: at first, it may seems that \texttt{EVCBoundSUKP} brings
little-to-no advantage compared to \texttt{EVCBoundScan}. We kindly ask the
reader to hold their judgment until the next section, where we modify the
algorithms presented here to take into account additional information available
on $\mathcal{C}$, and the practicality of \texttt{EVCBoundSUKP} will be more
evident.

\subsection{Computing an upper bound to $\EVC(\range(\mathcal{B}),\Ds)$} 
\label{sec:vcbupper}

Na\"ively, we
could use any of the algorithms presented in Sect.~\ref{sec:computvc} to compute
an upper bound to $\EVC(\range(\mathcal{F}),\Ds)$ and this would be a valid upper bound
to $\EVC(\range(\mathcal{B}),\Ds)$ (provided Fact~\ref{fact:supersetborder} holds). 

The
downside of this line of action is that $\mathcal{F}$ may be much larger than
$\mathcal{B}$ and, even more importantly, that we are choosing to ignore
available information on the structure of $\mathcal{B}$. In particular, we are
choosing to ignore the following.

\begin{fact}
	The negative border of a collection of itemsets is a \emph{maximal
	antichain} on $2^\Itm$.
\end{fact}

The collection $\mathcal{F}$ is in general not an antichain, except for
degenerate and unrealistic cases. An upper bound to $\EVC(\mathcal{F},\Ds)$
would be a gross overestimation of $\EVC(\mathcal{B},\Ds)$, hence we would like
to incorporate the fact that $\mathcal{B}$ is a maximal antichain in our
computation of the bound to its empirical VC-dimension.

We can modify \texttt{EVCBoundExact} by including, after having computed the
set $\mathcal{C}_\tau$ (line~\ref{algline:intersectionloopend} in
Alg.~\ref{alg:evcboundexact}), the size $\ell_\tau$ of the largest antichain in
$\mathcal{C}_\tau$, which can be done in time polynomial in $\mathcal{C}_\tau$.
We can then use $\ell_\tau$ as the value for $L[I_\tau]$
(line~\ref{algline:assignmentintersection}). Since $\ell_\tau$ is the size of
the largest antichain in $\mathcal{C}_\tau$, then $\ell_\tau$ is an upper bound
to the number of itemsets from $\mathcal{B}$ that appear in $\tau$. Provided
that $\Ds$ is a $\varepsilon_1$-approximation to $(\range(2^\Itm),\prob)$ (so
that Fact~\ref{fact:supersetborder} holds) the so-modified algorithm, which we
call \texttt{EVCBoundAntichain}, clearly gives an upper-bound to the d-index of
$\mathcal{B}$ on $\Ds$, and therefore to $\EVC(\mathcal{B},\Ds)$. Due to the
additional computation of $\ell_\tau$, this algorithm is slower than
$\texttt{EVCBoundExact}$. It is possible to modify the second part of
\texttt{EVCBoundAntiChain} in the same way as \texttt{EVCBoundExact} was
modified to obtain \texttt{EVCBoundScan}, and obtain a slightly faster
algorithm. Nevertheless, we believe that the need to compute potentially for
each transaction $\tau$ both the set of itemsets of $\mathcal{C}$ in $\tau$ and
additionally the size of the largest antichain on this set, make using
\texttt{EVCBoundAntichain} a bit unpractical, despite being a polynomial time
algorithm.

It is actually for this reason that we developed \texttt{EVCBoundSUKP}, which we
can modify \texttt{EVCBoundSUKP} to solve a slightly modified SUKPs associated
to $\mathcal{F}$ with the \emph{additional constraint} that the optimal solution
(a collection of itemsets) \emph{must be a maximal antichain}. The correctness
follows from the fact that the optimal profit of the so-modified SUKP with
capacity $\ell$ is the maximum number of itemsets forming an antichain in
$\mathcal{F}$ that can be built using a subset of $\ell$ of $U$, so it is an
upper bound to the number of itemsets in $\mathcal{B}$ that can be built on
$\ell$ items of $U$, provided that $\Ds$ is a $\varepsilon_1$-approximation to
$(\range(2^\Itm),\prob)$ (so that Fact~\ref{fact:supersetborder} holds). Hence
Lemma~\ref{lem:sukpevc} carries over and this new algorithm, called
\texttt{EVCBoundSUKPAntiChain} is correct. This algorithm is more practical than
\texttt{EVCBoundAntichain}, striking a more balanced trade-off between
exploiting information about $\mathcal{B}$ and obtaining an upper bound to its
empirical VC-dimension in a reasonable amount of time.