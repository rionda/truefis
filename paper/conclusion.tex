\section{Conclusions}\label{sec:concl}
The usefulness of frequent itemset mining is often hindered by spurious
discoveries, or false positives, in the results. In this work we developed an
algorithm to compute a frequency threshold $\hat\theta$ such that the collection
of FIs at frequency $\hat\theta$ is a good approximation the collection of True
Frequent Itemsets. The threshold is such that that the probability of reporting
\emph{any} false positive is bounded by a user-specified quantity. We used
concepts from statistical learning theory and from optimization to develop and
analyze the algorithm. The experimental evaluation shows that the method we
propose can indeed be used to control the presence of false positives while, at
the same time, extracting a very large fraction of the TFIs from huge datasets.

There are a number of directions for further research. Among these, we find
particularly interesting and challenging the extension of our method to other
definitions of statistical significance for patterns and to other definitions of
patterns such as sequential patterns~\citep{LowCamRKP13}. Also interesting is
the derivation of better lower bounds to the VC-dimension of the range set of a
collection of itemsets. Moreover, while this work focuses on itemsets mining, we
believe that it can be extended and generalized to other settings of multiple
hypothesis testing, and give another alternative to existing approaches for
controlling the probability of false discoveries.
