\section{The range set of a collection of itemsets}\label{sec:range}
In this section we define the concept of a range set associated to a
collection of itemsets and show how to bound the VC-dimension and the
empirical VC-dimension of this range set. We use these definitions and results
to develop our algorithm in Sect.~\ref{sec:main}.

\begin{definition}\label{def:rangeset}
Given a collection $\mathcal{C}$ of itemsets built on a ground set $\Itm$, the
\emph{range set $\range(\mathcal{C})$ associated to $\mathcal{C}$ is a range
set on $2^\Itm$} containing the support sets of the itemsets in $\mathcal{C}$:
\[
	\range(\mathcal{C})=\{T(A) ~:~ A\in\mathcal{C}\}\enspace.
\]
\end{definition}

\begin{fact}\label{fact:maxfreq}
	Given a collection $\mathcal{C}$ of itemsets and a dataset $\Ds$, the maximum
	in~\eqref{eq:evceapprox} is attained for the itemset $A\in\mathcal{C}$ with the
	highest frequency in $\Ds$, and the value of $\sqrt{|r\cap S|}$ is exactly
	$\sqrt{f_\Ds(A)|\Ds|}$. Hence we can rewrite~\eqref{eq:evceapprox} as
	\begin{equation}\label{eq:evceapproxitemsets}
		\varepsilon = 2c\sqrt{\frac{2d\max_{A\in\mathcal{C}}f_\Ds(A)}{|\Ds|}} +
		\sqrt{\frac{2\ln\frac{4}{\delta}}{|\Ds|}}\enspace.
		% THIS IS HOW THE EQUATION USED TO BE BEFORE THE NEW BOUND
		%2\sqrt{\frac{\max_{A\in\mathcal{C}}f_\Ds(A)2\min\left\{\ln|\mathcal{C}|,d\ln\frac{en}{d}\right\}}{|\Ds|}} +
		%\sqrt{\frac{2\ln\frac{4}{\delta}}{|\Ds|}},
	\end{equation}
	where $d$ is an
	upper bound to $\EVC(\range(\mathcal{C}),\Ds)$, and $\range(\mathcal{C})$ is
	as in Def.~\ref{def:rangeset}.
\end{fact}

The following Theorem presents an upper bound to the empirical VC-dimension of
$\range(\mathcal{C})$ on a dataset $\Ds$.

\begin{theorem}\label{lem:evcdimupbound}
  Let $\mathcal{C}$ be a collection of itemsets, $\Ds$ be a dataset, and $U$ be
  the set of items appearing in the itemsets of $\mathcal{C}$ (i.e., $U=\{a\in A ~:~
  A\in\mathcal{C}\}$). Let $d$ be the maximum integer for which there are at
  least $d$ transactions $\tau_1,\dotsc,\tau_d\in \Ds$ such that:
  \begin{itemize}
	\item the set $\{\tau_1\cap U,\dotsc,\tau_d\cap U\}$ is an antichain; and
	\item each $\tau_i$, $1\le i\le d$, contains at least $2^{d-1}$ itemsets
		from $\mathcal{C}$;
  \end{itemize}
  Then $\EVC(\range(\mathcal{C}),\Ds)\le d$.
\end{theorem}

\begin{proof}
  The antichain requirement guarantees that the set of transactions considered in
  the computation of $d$ could indeed theoretically be shattered. Assume that a
  subset $\mathcal{F}$ of $\Ds$ contains two transactions $\tau'$ and $\tau''$
  such that, w.l.o.g., $\tau' \cap U\subseteq\tau''\cap U$. Any itemset from
  $\mathcal{C}$ appearing in $\tau'$ would also appear in $\tau''$, so there
  would not be any itemset $A\in\mathcal{C}$ such that $\tau''\in T(A)\cap
  \mathcal{F}$ but $\tau'\not\in T(A)\cap \mathcal{F}$, which would imply that
  $\mathcal{F}$ can not be shattered. Hence sets of transactions that are not
  antichains should not be considered.
  %This has the net effect of potentially resulting in a lower $d$,
  %i.e., in a stricter upper bound to $\EVC(\range(\mathcal{C}),\Ds)$.

  Let now $\ell>d$ and consider a set $\mathcal{L}$ of $\ell$ transactions from
  $\Ds$ that satisfies the antichain requirement. Assume that $\mathcal{L}$ is
  shattered by $\range(\mathcal{C})$. Let $\tau$ be a transaction in
  $\mathcal{L}$. The transaction $\tau$ belongs to $2^{\ell-1}$ subsets of $\mathcal{L}$.
  Let $\mathcal{K}\subseteq \mathcal{L}$ be one of these subsets. Since
  $\mathcal{L}$ is shattered, there exists an itemset $A\in\mathcal{C}$ such
  that $T(A)\cap \mathcal{L}=\mathcal{K}$. From this and the fact that $\tau\in
  \mathcal{K}$, we have that $\tau\in T(A)$ or equivalently that
  $A\subseteq\tau$. Given that all the subsets $\mathcal{K}\subseteq\mathcal{L}$
  containing $\tau$ are different, then also all the $T(A)$'s such that
  $T(A)\cap \mathcal{L}=\mathcal{K}$ should be different, which in turn implies
  that all the itemsets $A$ should be different and that they should all appear
  in $\tau$. There are $2^{\ell-1}$ subsets $\mathcal{K}$ of $\mathcal{L}$
  containing $\tau$, therefore $\tau$ must contain at least $2^{\ell-1}$
  itemsets from $\mathcal{C}$, and this holds for all $\ell$ transactions in
  $\mathcal{L}$. This is a contradiction because $\ell>d$ and $d$ is the maximum
  integer for which there are at least $d$ transactions containing at least
  $2^{d-1}$ itemsets from $\mathcal{C}$. Hence $\mathcal{L}$ cannot be shattered
  and the thesis follows.
\end{proof}

Generalizing a concept by~\citet{RiondatoU14}, we call $d$ as defined in
Thm.~\ref{lem:evcdimupbound} the \emph{d-index of $\mathcal{C}$ in $\Ds$} \fabio{[Maybe we can find a more informative name?]}.

The antichain requirement from Thm.~\ref{lem:evcdimupbound} implies that, in the
set that gives origin to the d-index, there is no $\tau_i$, $1\le i\le d$ such
that $U\subseteq\tau_i$ (i.e., no $\tau_i$ for which $\tau_i\cap U=U$).

Instead of $\mathcal{C}$, one can consider a filtered version
$\mathcal{C}^*$ of $\mathcal{C}$ containing only the itemsets of $\mathcal{C}$
that are \emph{closed} in $\Ds$, i.e., for which no superset of them has the same
frequency in $\Ds$~\citep{CaldersRB06}. From the definitions of the ranges and
considering their intersections with $\Ds$, it is easy to see that the empirical
VC-dimension of $\mathcal{C}^*$ on $\Ds$ is the same as the empirical
VC-dimension of $\mathcal{C}$ on $\Ds$. Unless one already has access to the
frequencies of the itemsets in $\mathcal{C}$, this filtering operation would
require mining the dataset to obtain such frequencies. For clarity of
presentation, we will not consider the filtering step in the following
sections, but the reader should keep in mind that it can be applied whenever the
frequencies of the itemsets in $\mathcal{C}$ are available.

Mutuating a construction by~\citet{RiondatoU14}, one can show that the bound in
Thm.~\ref{lem:evcdimupbound} is tight, in the sense that there are datasets
such that the empirical VC-dimension of $\mathcal{C}$ on them is equal to the
d-index of $\mathcal{C}$ on them.
\fabio{[Add a Proposition stating this result?]}

