\section{The range set of a collection of itemsets}\label{sec:range}
In this section we define the concept of a range set associated to a
collection of itemsets and show how to bound the VC-dimension and the
empirical VC-dimension of this range set. We use these definitions and results
to develop our algorithm in later sections.

\begin{definition}\label{def:rangeset}
Given a collection $\mathcal{C}$ of itemsets built on a ground set $\Itm$, the
\emph{range set $\range(\mathcal{C})$ associated to $\mathcal{C}$ is a range
set on $2^\Itm$} containing the support sets of the itemsets in $\mathcal{C}$:
\[
	\range(\mathcal{C})=\{T(A) ~:~ A\in\mathcal{C}\}\enspace.
\]
\end{definition}

\begin{fact}\label{fact:maxfreq}
	Given this definition and a dataset $\Ds$, the maximum
	in~\eqref{eq:evceapprox} is attained for the itemset $A\in\mathcal{C}$ with the
	highest frequency in $\Ds$, and the value of $\sqrt{|r\cap S|}$ is exactly
	$\sqrt{f_\Ds(A)|\Ds|}$. Hence we can rewrite~\eqref{eq:evceapprox} as
	\begin{equation}\label{eq:evceapproxitemsets}
		\varepsilon =
		2\sqrt{\frac{\max_{A\in\mathcal{C}}f_\Ds(A)2\min\left\{\ln|\mathcal{C}|,d\ln\frac{en}{d}\right\}}{|\Ds|}} +
		\sqrt{\frac{2\ln\frac{4}{\delta}}{|\Ds|}},
	\end{equation}
	where $\mathcal{C}$ is as in Def.~\ref{def:rangeset}, and $d$ is an
	upper bound to $\EVC(\range(\mathcal{C}),\Ds)$.
\end{fact}

The following Theorem presents an upper bound to the empirical VC-dimension of
$\range(\mathcal{C})$ on a dataset $\Ds$.

\begin{theorem}\label{lem:evcdimupbound}
  Let $\mathcal{C}$ be a collection of itemsets and let $\Ds$ be a dataset. Let
  $d$ be the maximum integer for which there are at least $d$
  transactions $\tau_1,\dotsc,\tau_d\in \Ds$ such that the set
  $\{\tau_1,\dotsc,\tau_d\}$ is an \emph{antichain}\footnote{An antichain is a
  collection of sets such no one of them is a subset of another.}, and each $\tau_i$, $1\le i\le d$,
  contains at least $2^{d-1}$ itemsets from $\mathcal{C}$.
  %\begin{enumerate*}
  %  \item
  %    The set $\{\tau_1,\dotsc,\tau_d\}$ is an antichain, and
  %    %For any pair of transactions $(\tau_i,\tau_j)$, $i\neq j$, we have
  %    %$\tau_i\not\subseteq\tau_j$ and $\tau_j\not\subseteq\tau_i$, and
  %  \item Each $\tau_i$, $1\le i\le d$ contains at least $2^{d-1}$ itemsets from
  %    $\mathcal{C}$.
  %\end{enumerate*}
  Then $\EVC(\range(\mathcal{C}),\Ds)\le d$.
\end{theorem}

\begin{proof}
  The antichain requirement guarantees that the set of transactions considered in
  the computation of $d$ could indeed theoretically be shattered. Assume that a
  subset $\mathcal{F}$ of $\Ds$ contains two transactions $\tau'$ and $\tau''$
  such that $\tau'\subseteq\tau''$. Any itemset from $\mathcal{C}$
  appearing in $\tau'$ would also appear in $\tau''$, so there would not be any
  itemset $A\in\mathcal{C}$ such that $\tau''\in T(A)\cap F$ but
  $\tau'\not\in T(A)\cap \mathcal{F}$, which would imply that $\mathcal{F}$ can
  not be shattered. Hence sets that are not antichains should not be
  considered. This has the net effect of potentially resulting in a lower $d$,
  i.e., in a stricter upper bound to $\EVC(\range(\mathcal{C}),\Ds)$.

  Let now $\ell>d$ and consider a set $\mathcal{L}$ of $\ell$ transactions from
  $\Ds$ that is an antichain. Assume that $\mathcal{L}$ is shattered by
  $\range(\mathcal{C})$. Let $\tau$ be a transaction in $\mathcal{L}$.  The
  transactions $\tau$ belongs to $2^{\ell-1}$ subsets of $L$. Let
  $\mathcal{K}\subseteq \mathcal{L}$ be one of these subsets. Since
  $\mathcal{L}$ is shattered, there exists an itemset $A\in\mathcal{C}$ such
  that $T(A)\cap \mathcal{L}=\mathcal{K}$. From this and the fact that $t\in
  \mathcal{K}$, we have that $\tau\in T(A)$ or equivalently that
  $A\subseteq\tau$. Given that all the subsets $\mathcal{K}\subseteq\mathcal{L}$
  containing $\tau$ are different, then also all the $T(A)$'s such that
  $T(A)\cap \mathcal{L}=\mathcal{K}$ should be different, which in turn implies
  that all the itemsets $A$ should be different and that they should all appear
  in $\tau$. There are $2^{\ell-1}$ subsets $\mathcal{K}$ of $\mathcal{L}$
  containing $\tau$, therefore $\tau$ must contain at least $2^{\ell-1}$
  itemsets from $\mathcal{C}$, and this holds for all $\ell$ transactions in
  $\mathcal{L}$. This is a contradiction because $\ell>d$ and $d$ is the maximum
  integer for which there are at least $d$ transactions containing at least
  $2^{d-1}$ itemsets from $\mathcal{C}$. Hence $\mathcal{L}$ cannot be shattered
  and the thesis follows.
\end{proof}

\subsection{Computing the VC-Dimension}\label{sec:computvc}
The na\"ive computation of $d$  according to the definition in
Thm.~\ref{lem:evcdimupbound} requires to scan the transactions one by one,
compute the number of itemsets from $\mathcal{C}$ appearing in each transaction,
and make sure to consider only itemsets constituting antichains. Given the very
large number of transactions in typical dataset and the fact that the number of
itemsets in a transaction is exponential in its length, this method would be
computationally too expensive. An upper bound to $d$ (and therefore to
$\EVC(\range(\mathcal{C}),\Ds)$) can be computed by solving a \emph{Set-Union
Knapsack Problem} (SUKP)~\citep{GoldschmidtNY94} associated to $\mathcal{C}$.

\begin{definition}[Set Union Knapsack Problem (SUKP)\citep{GoldschmidtNY94}]\label{def:sukp}
  Let $U=\{a_1,\dotsc,a_\ell\}$ be a set of elements and let
  $\mathcal{S}=\{A_1,\dotsc,A_k\}$ be a set of subsets of $U$, i.e.,
  $A_i\subseteq U$ for $1\le i\le k$. Each subset $A_i$, $1\le i\le k$, has an associated
  non-negative \emph{profit} $\rho(A_i)\in\mathbb{R}^+$, and each element $a_j$, $1\le
  j\le\ell$ as an associated non-negative weight $w(a_j)\in\mathbb{R}^+$.
  Given a subset $\mathcal{S}'\subseteq\mathcal{S}$, we define the profit of
  $\mathcal{S}'$ as $P(\mathcal{S}')=\sum_{A_i\in \mathcal{S}'}\rho(A_i)$. Let
  $U_{\mathcal{S}'}=\cup_{A_i\in\mathcal{S}'} A_i$. We
  define the weight of $\mathcal{S}'$ as $W(\mathcal{S}')=\sum_{a_j\in
  U_{\mathcal{S}'}} w(a_j)$. Given a non-negative parameter $c$ that we call
  \emph{capacity}, the \emph{Set-Union Knapsack Problem} (SUKP) requires to find
  the set $\mathcal{S}^*\subseteq\mathcal{S}$ which \emph{maximizes}
  $P(\mathcal{S}')$ over all sets $\mathcal{S}'$ such that $W(\mathcal{S}')\le c$.
\end{definition}

In our case, $U$ is the set of items that appear in the itemsets of
$\mathcal{C}$, $\mathcal{S}=\mathcal{C}$, the profits and the weights are all
unitary, and the capacity constraint is an integer $\ell$. We call this
optimization problem the \emph{SUKP associated to $\mathcal{C}$ with capacity
$\ell$}. It is easy to see that the optimal profit of this SUKP is the maximum
number of itemsets from $\mathcal{C}$ that a transaction of length $\ell$ can
contain. In order to show how to use this fact to compute an upper bound to
$\EVC(\range(\mathcal{C}),\Ds)$, we need to define some additional terminology.
Let $\ell_1,\dotsc,\ell_w$ be the sequence of the \emph{transaction lengths} of
$\Ds$, i.e., for each value $\ell$ for which there is at least a transaction in
$\Ds$ of length $\ell$, there is one (and only one) index $i$, $1\le i\le w$
such that $\ell_i=\ell$. Assume that the $\ell_i$'s are labelled in sorted
decreasing order: $\ell_1>\ell_2>\dotsb>\ell_w$. Let now $L_i$, $1\le i\le w$ be
the maximum number of transactions in $\Ds$ that have length at least $\ell_i$
and such that for no two $\tau'$, $\tau''$ of them we have either
$\tau'\subseteq\tau''$ or $\tau''\subseteq\tau'$. Let now $q_i$ be the optimal
profit of the SUKP associated to $\mathcal{C}$ with capacity $L_i$, and let
$b_i=\lfloor \log_2q_i\rfloor +1$.  The sequences $(\ell_i)_1^w$ and a sequence
$(L_i^*)^w$ of upper bounds to $(L_i)_1^w$ can be computed efficiently with a
scan of the dataset.  The following lemma uses these sequences to show how to
obtain an upper bound to the empirical VC-dimension of $\mathcal{C}$ on $\Ds$.

\begin{lemma}\label{lem:sukpevc}
  Let $j$ be the minimum integer for which $b_i\le L_i$. Then
  $\EVC(\mathcal{C},\Ds)\le b_j$.
\end{lemma}
\begin{proof}
  If $b_j\le L_j$, then there are at least $b_j$ transactions which can contain
  $2^{b_j-1}$ itemsets from $\mathcal{C}$ and this is the maximum $b_i$ for
  which it happens, because the sequence $b_1,b_2,\dotsc,b_w$ is sorted in
  decreasing order, given that the sequence $q_1,q_2,\dotsc,q_w$ is. Then $b_j$
  satisfies the conditions of Lemma~\ref{lem:evcdimupbound}. Hence
  $\EVC(\mathcal{C},\Ds)\le b_j$.
\end{proof}

% The following is trivially true: the solution to this SUKP is the whole
% collection of sets!
%\begin{corollary}\label{lem:sukpvc}
%  Let $q$ be profit of the SUKP associated to $\mathcal{C}$ with capacity
%  equal to $\ell=|\{a\in\Itm ~:~ \exists A\in\mathcal{C} \mbox{ s.t. } a\in
%  A\}|$ ($\ell$ is the number of items such that there is at least one itemset
%  in $\mathcal{C}$ containing  them).
%  %$|\Itm|-1$.
%  Let $b=\lfloor\log_2 q\rfloor + 1$. Then
%  $\VC(\range(\mathcal{C}))\le b$. %We call $b*$ the \emph{b-index of
%  %$\mathcal{C}$} and denote it as $\b(\mathcal{C})$.
%\end{corollary}

\paragraph{Complexity and runtime considerations.} Solving the SUKP optimally is
NP-hard in the general case, although there are known restrictions for which it
can be solved in polynomial time using dynamic
programming~\citep{GoldschmidtNY94}. Since we have unit weights and unit
profits, our SUKP is equivalent to the \emph{densest $k$-subhypergraph} problem,
which can not be approximated within a factor of $2^{O(\log n)^\delta}$ for any
$\delta>0$ unless $3STA \in
DTIME(2^{n^{3/4+\varepsilon}})$~\citep{HajiaghayiJKLMRSV06}. A greedy algorithm
by~\citet{Arulselvan14} allows a constant factor approximation if each
items only appear in a constant fraction of itemsets of $\mathcal{C}$. For our
case, it is actually \emph{not necessary to compute the optimal solution} to the
SUKP: any upper bound solution for which we can prove that there is no power of
two between that solution and the optimal solution would result in the
\emph{same upper bound} to the (empirical) VC-dimension, while substantially
speeding up the computation. This property can be specified in currently
available optimization problem solvers (e.g., CPLEX), which can then can compute
the bound to the (empirical) VC-dimension very fast even for very large
instances with thousands of items and hundred of thousands of itemsets in
$\mathcal{C}$, making this approach practical.

\paragraph{Refinements.} It is possible to make some refinements to our
computation of the \emph{empirical} VC-dimension of a collection $\mathcal{C}$ of
itemsets on a dataset $\Ds$. First of all, one can remove from $\mathcal{C}$ all
itemsets that never appear in $\Ds$, as the corresponding ranges can not help
shattering any set of transactions in $\Ds$. Identifying which itemsets to
remove requires a single linear scan of $\Ds$. Secondly, when computing the
capacities $L_i$ (i.e., their upper bounds $L_i^*$), we can ignore all the
transactions that do not contain \emph{any} of the itemsets in $\mathcal{C}$ (or
the filtered version of $\mathcal{C}$), as there is no way of shatter them using
the ranges corresponding to itemsets in $\mathcal{C}$. Both refinements aim at
reducing the optimal value of the SUKP associated to $\mathcal{C}$, and
therefore at computing a smaller bound to the empirical VC-dimension of
$\mathcal{C}$ on $\Ds$. We remark that these refinements can not be used when
computing the (non-empirical) VC-dimension.

\paragraph{The range set of all itemsets.}
The range set associated to $2^\Itm$ is particularly interesting for us. It is
possible to compute bounds to $\VC(\range(2^\Itm))$ and $\EVC(\range(2^\Itm),
\Ds)$ without having to solve a SUKP.
\begin{theorem}[\citep{RiondatoU14}]\label{thm:empvcdimubfirst}
  Let $\Ds$ be a dataset built on a
  ground set $\Itm$. The
  \emph{d-index} $\mathsf{d}(\Ds)$ of $\Ds$ is the maximum integer $d$ such that
  $\Ds$ contains at least $d$ transactions of length at least $d$ that form an
  antichain.

  We have $\EVC(\range(2^\Itm),\Ds)\le \mathsf{d}(\Ds)$.
\end{theorem}

\begin{corollary}\label{thm:vcdimubfirst}
  $\VC(\range(2^\Itm))\le |\Itm|-1$.
\end{corollary}

\citet{RiondatoU14} presented an efficient algorithm to compute an upper bound to
the d-index of a dataset with a single linear scan of the dataset $\Ds$.
The upper bound presented in Thm.~\ref{thm:empvcdimubfirst} is tight: there are
datasets for which
$\EVC(\range(2^\Itm),\Ds)=\mathsf{d}(\Ds)$~\citep{RiondatoU14}.  This implies
that the upper bound presented in Corol.~\ref{thm:vcdimubfirst} is
also tight.
