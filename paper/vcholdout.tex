\subsection{Method 1: a split-dataset approach}\label{sec:holdout}
%\XXX: need better title. \MR
%\FV{What about ``a partition-based approach''?}

We now present and analyze our first method for identifying TFIs, which draws
inspiration from the holdout technique presented by~\citet{Webb07}. This test
requires that the dataset $\Ds$ could be randomly split into two
parts, not necessarily of the same size: an \emph{exploratory} part
$\Ds_\mathrm{e}$ and an \emph{evaluation} part $\Ds_\mathrm{v}$. 
%The method does not require the sizes $|\Ds_\mathrm{e}|$ and $|\Ds_\mathrm{v}|$ to be the same.
The method works in two phases: we first use the exploratory part
$\Ds_\mathrm{e}$ to identify a small set $\mathcal{G}$ of candidate TFIs which will be used to
compute the acceptance region, then we test these same itemsets using their
frequencies in the evaluation part $\Ds_\mathrm{v}$ as test statistics.

Let $\delta_\mathrm{e}$ and $\delta_\mathrm{v}$ be such that
$(1-\delta_\mathrm{e})(1-\delta_\mathrm{v})\ge(1-\delta)$. Let $\range(2^\Itm)$
be the range space of all itemsets.
We use Corol.~\ref{thm:vcdimubfirst} (resp.~Thm.~\ref{thm:empvcdimubfirst}) to
compute an upper bound $d'$ to $\VC(\range(2^\Itm))$ (resp.~ $d''$ to
$\EVC(\range(2^\Itm),\Ds_\mathrm{e})$). Then, given that $\Ds_\mathrm{e}$ is
still a collection of i.i.d.~samples from the generative distribution $\prob$,
we can use $d'$ in Thm.~\ref{thm:eapprox} (resp.~$d''$ in
Thm~\ref{thm:eapproxempir}) to compute an $\varepsilon_\mathrm{e}'$ (resp.~an
$\varepsilon_\mathrm{e}''$) such that $\Ds_\mathrm{e}$ is, with probability at
least $1-\delta_1$, an $\varepsilon_\mathrm{e}'$-approximation
(resp.~$\varepsilon_\mathrm{e}''$-approximation) to $(\range(2^\Itm),\prob)$.
Then, if we let
$\varepsilon_\mathrm{e}=\min\{\varepsilon_\mathrm{e}',\varepsilon_\mathrm{e}''\}$,
$\Ds_\mathrm{e}$ is, with probability at least $1-\delta_1$, an
$\varepsilon_\mathrm{e}$-approximation to $(\range(2^\Itm),\prob)$.

In the first phase of the set we compute the collections of itemsets
$\mathcal{C}_\mathrm{e}=\{X\subseteq\Itm ~:~
f_{\Ds_\mathrm{e}}(X)\ge\theta+\varepsilon_\mathrm{e}\}$
and $\mathcal{G}=\{X\subseteq\Itm ~:~ \theta\le
f_{\Ds_\mathrm{e}}(X)<\theta+\varepsilon_\mathrm{e}\}$. To obtain these sets, we
extract the set $\FI(\Ds_\mathrm{e},\Itm,\theta)$ and partition it appropriately
into $\mathcal{C}_\mathrm{e}$ and $\mathcal{G}$.
In the second phase, we compute a value $\varepsilon_\mathrm{v}$ such that, with
probability at least $1-\delta_2$, the evaluation dataset $\Ds_\mathrm{v}$ is an
$\varepsilon_\mathrm{v}$-approximation to $(\range(\mathcal{G}),\prob)$. In
order to obtain $\varepsilon_\mathrm{v}$ through 
Thms.~\ref{thm:eapprox} and~\ref{thm:eapproxempir}, we need to compute upper
bounds to $\VC(\range(\mathcal{G}))$ and
$\EVC(\range(\mathcal{G}),\Ds_\mathrm{v})$. We solve SUKPs assocociated to
$\mathcal{G}$ and obtain such bounds, as stated in Corol.~\ref{lem:sukpvc} and
Lemma~\ref{lem:sukpevc} respectively. As in the
first phase, we use these bounds and Thm.~\ref{thm:eapprox}
(resp.~Thm.~\ref{thm:eapproxempir}) to compute a
$\varepsilon_\mathrm{v}'$ (resp.~ $\varepsilon_\mathrm{v}''$) such that
$\Ds_\mathrm{v}$ is, with probability at least $1-\delta_2$, an
$\varepsilon_\mathrm{v}'$-approximation (resp.~an
$\varepsilon_\mathrm{v}''$-approximation) to $(\range(\mathcal{G}),\prob)$. Once
we have obtained
$\varepsilon_\mathrm{v}=\min\{\varepsilon_\mathrm{v}',\varepsilon_\mathrm{v}''\}$,
we compute the set
\[
\mathcal{C}_\mathrm{v}=\{X\subseteq\Itm ~:~ X\in\mathcal{G} \mbox{ and }
f_{\Ds_\mathrm{v}}(X)\ge\theta+\varepsilon''\}=\mathcal{G}\cap\FI(\Ds_\mathrm{v},\Itm,\theta+\varepsilon_\mathrm{v}).\]
The method returns the collection of itemsets $\mathcal{C}_\mathrm{e}\cup\mathcal{C}_\mathrm{v}$.
%The correctness of this test, i.e., 
The proof that $\Pr(\exists
A\in\mathcal{C}_\mathrm{e}\cup\mathcal{C}_\mathrm{v} \mbox{ s.t. }
\tfreq(A)<\theta)\le\delta$ comes from the definition of $\varepsilon_\mathrm{e}$ and
$\varepsilon_\mathrm{v}$ through Thms.~\ref{thm:eapprox}
and~\ref{thm:eapproxempir} (proof in the full version
of the paper~\citep{RiondatoV13}).
%Using Thm. \FV{which ones?}, we prove that the probability that $\mathcal{C}_\mathrm{e}\cup\mathcal{C}_\mathrm{v}$ contains any itemset not in TFI's
%is bounded by $\delta$.

%\XXX: pseudocode? \MR

\begin{lemma}
  The FWER of Method 1 is at most $\delta$.
\end{lemma}
\ifarxiv
\begin{proof}
  Consider the two events $\mathsf{E}_\mathrm{e}$=``$\Ds_\mathrm{e}$ is an
  $\varepsilon_\mathrm{e}$-approximation for $(\range(2^\Itm),\prob)$'' and
  $\mathsf{E}_\mathrm{v}=$``$\Ds_\mathrm{v}$ is an
  $\varepsilon_\mathrm{v}$-approximation for $(\range(\mathcal{G}),\prob)$. From
  the above discussion it follows that the event
  $\mathsf{E}=\mathsf{E}_\mathrm{e}\cap\mathsf{E}_\mathrm{v}$ occurs with probability at least
  $1-\delta$. Suppose from now on that indeed $\mathsf{E}$ occurs.

  Given that $\mathsf{E}_\mathrm{e}$ occurs, then all the itemsets with
  frequency in $\Ds_\mathrm{e}$ at least $\theta+\varepsilon_\mathrm{e}$
  must have a real frequency at least $\theta$. This equals to say that all
  itemsets in $\mathcal{C}_\mathrm{e}$ are True Frequent Itemsets
  ($\mathcal{C}_\mathrm{e}\subseteq TFI(\theta)$.

  Given that $\mathsf{E}_\mathrm{v}$ occurs, then we know that all itemsets in
  $\mathcal{G}$ have frequency in $\Ds_\mathrm{v}$ that is at most
  $\varepsilon_\mathrm{v}$ far from their true frequency:
  \[
  \max_{A\in\mathcal{G}}\left|\tfreq(A)-f_{\Ds_\mathrm{v}}(A)\right|\le\varepsilon_\mathrm{v}\enspace.\]
  In particular this means that an  itemset $A\in\mathcal{G}$ can have
  $f_{Ds_\mathrm{v}}(A)\ge\theta+\varepsilon_\mathrm{v}$ \emph{only} if
  $\tfreq(A)\ge\theta$, that is \emph{only} if $A\in\TFI(\prob,\Itm,\theta)$.
  Hence, $\mathcal{C}_\mathrm{v}\subseteq TFI(\prob,\Itm,\theta)$. 

  We can then conclude that if the event $\mathsf{E}$ occurs, we have
  $\mathcal{C}_\mathrm{e}\cup \mathcal{C}_\mathrm{v}\subseteq TFI(\prob,\Itm,\theta)$.
  Since $\mathsf{E}$ occurs with probability at least $1-\delta$, this equals to
  say that 
  \[
  \Pr(\exists A\in \mathcal{T}_\mathrm{e}\cup
  \mathcal{T}_\mathrm{v} ~:~ \tfreq(A)<\theta)\le\delta\enspace.\]
  \qed
\end{proof}
\fi
