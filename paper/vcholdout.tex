\subsection{An holdout approach}\label{sec:holdout}

{\bf PER MATTEO: } Sistema questa sezione dopo gli ultimi stravolgimenti

We now present and analyze a second method, called \ALGHOLDOUT, for
mining the TFIs. \ALGHOLDOUT draws inspiration from the holdout technique
presented by~\citet{Webb07}, and requires that the dataset $\Ds$ can be
randomly split into two parts that do not need to be of the same size, but
it \emph{must} be possible to see them as two independent collections of
i.i.d.~samples from the generating distribution $\pi$: an \emph{exploratory}
part $\Ds_\mathrm{e}$ and an \emph{evaluation} part $\Ds_\mathrm{v}$.
\ALGHOLDOUT works in two phases: we first use the exploratory part
$\Ds_\mathrm{e}$ to identify a small set $\mathcal{G}$ of candidate TFIs, and
then we decide which of the candidates to include in the output using their
frequencies in the evaluation part $\Ds_\mathrm{v}$.

\paragraph{First phase.} Let $\delta_\mathrm{e}$ and $\delta_\mathrm{v}$ be such
that $(1-\delta_\mathrm{e})(1-\delta_\mathrm{v})\ge(1-\delta)$. Let
$\range(2^\Itm)$ be the range space of all itemsets. We can compute an upper
bound $d'$ to $\VC(\range(2^\Itm))$ and an upper bound $d''$ to
$\EVC(\range(2^\Itm),\Ds_\mathrm{e})$, using respectively
Corol.~\ref{thm:vcdimubfirst} and Thm.~\ref{thm:empvcdimubfirst}). Then, given
that $\Ds_\mathrm{e}$ is a collection of i.i.d.~samples from the generative
distribution $\prob$, we can use $d'$ in Thm.~\ref{thm:eapprox} (resp.~$d''$ in
Thm~\ref{thm:eapproxempir}) to compute an $\varepsilon_\mathrm{e}'$ (resp.~an
$\varepsilon_\mathrm{e}''$) such that $\Ds_\mathrm{e}$ is, with probability at
least $1-\delta_1$, an $\varepsilon_\mathrm{e}'$-approximation
(resp.~$\varepsilon_\mathrm{e}''$-approximation) to $(\range(2^\Itm),\prob)$.
Then, if we let
$\varepsilon_\mathrm{e}=\min\{\varepsilon_\mathrm{e}',\varepsilon_\mathrm{e}''\}$,
we have that $\Ds_\mathrm{e}$ is, with probability at least $1-\delta_1$, an
$\varepsilon_\mathrm{e}$-approximation to $(\range(2^\Itm),\prob)$. These steps
are exactly the same as the first steps of \ALG, which operated on the
\emph{whole} dataset, not just on one part like \ALGHOLDOUT. We then compute the
collections of itemsets $\mathcal{C}_\mathrm{e}=\{X\subseteq\Itm ~:~
f_{\Ds_\mathrm{e}}(X)\ge\theta+\varepsilon_\mathrm{e}\}$ and
$\mathcal{G}=\{X\subseteq\Itm ~:~ \theta\le
f_{\Ds_\mathrm{e}}(X)<\theta+\varepsilon_\mathrm{e}\}$. To obtain these sets, we
extract the set $\FI(\Ds_\mathrm{e},\Itm,\theta)$ and partition it appropriately
into $\mathcal{C}_\mathrm{e}$ and $\mathcal{G}$.

\paragraph{Second phase.} In the second phase, we compute a value
$\varepsilon_\mathrm{v}$ such that, with probability at least $1-\delta_2$, the
evaluation dataset $\Ds_\mathrm{v}$ is an $\varepsilon_\mathrm{v}$-approximation
to $(\range(\mathcal{G}),\prob)$. In
order to obtain $\varepsilon_\mathrm{v}$ through
Thms.~\ref{thm:eapprox} and~\ref{thm:eapproxempir}, we need to compute upper
bounds to $\VC(\range(\mathcal{G}))$ and
$\EVC(\range(\mathcal{G}),\Ds_\mathrm{v})$. We solve SUKPs associated to
$\mathcal{G}$ and obtain such bounds, as stated in Lemma~\ref{lem:sukpevc}
respectively. We then use these bounds and Thm.~\ref{thm:eapprox}
(resp.~Thm.~\ref{thm:eapproxempir}) to compute a
$\varepsilon_\mathrm{v}'$ (resp.~ $\varepsilon_\mathrm{v}''$) such that
$\Ds_\mathrm{v}$ is, with probability at least $1-\delta_2$, an
$\varepsilon_\mathrm{v}'$-approximation (resp.~an
$\varepsilon_\mathrm{v}''$-approximation) to $(\range(\mathcal{G}),\prob)$. Once
we have obtained
$\varepsilon_\mathrm{v}=\min\{\varepsilon_\mathrm{v}',\varepsilon_\mathrm{v}''\}$,
we compute the set
\[
\mathcal{C}_\mathrm{v}=\{X\subseteq\Itm ~:~ X\in\mathcal{G} \mbox{ and }
f_{\Ds_\mathrm{v}}(X)\ge\theta+\varepsilon''\}=\mathcal{G}\cap\FI(\Ds_\mathrm{v},\Itm,\theta+\varepsilon_\mathrm{v}).\]
The method returns the collection of itemsets $\mathcal{C}_\mathrm{e}\cup\mathcal{C}_\mathrm{v}$.
%The proof that $\Pr(\exists
%A\in\mathcal{C}_\mathrm{e}\cup\mathcal{C}_\mathrm{v} \mbox{ s.t. }
%\tfreq(A)<\theta)\le\delta$ comes from the definition of $\varepsilon_\mathrm{e}$ and
%$\varepsilon_\mathrm{v}$ through Thms.~\ref{thm:eapprox}
%and~\ref{thm:eapproxempir}

\begin{theorem}
	With probability at least $1-\delta$,
	$\mathcal{C}_\mathrm{e}\cup\mathcal{C}_\mathrm{v}$ contains no false
	positives:
	\[
		\Pr(\exists A\in\mathcal{C}_\mathrm{e}\cup\mathcal{C}_\mathrm{v} \mbox{
		s.t. } \tfreq(A)<\theta)\le\delta\enspace.
	\]
\end{theorem}
\begin{proof}
  Consider the two events $\mathsf{E}_\mathrm{e}$=``$\Ds_\mathrm{e}$ is an
  $\varepsilon_\mathrm{e}$-approximation for $(\range(2^\Itm),\prob)$'' and
  $\mathsf{E}_\mathrm{v}=$``$\Ds_\mathrm{v}$ is an
  $\varepsilon_\mathrm{v}$-approximation for $(\range(\mathcal{G}),\prob)$. From
  the above discussion it follows that the event
  $\mathsf{E}=\mathsf{E}_\mathrm{e}\cap\mathsf{E}_\mathrm{v}$ occurs with probability at least
  $1-\delta$. Suppose from now on that indeed $\mathsf{E}$ occurs.

  Given that $\mathsf{E}_\mathrm{e}$ occurs, then all the itemsets with
  frequency in $\Ds_\mathrm{e}$ at least $\theta+\varepsilon_\mathrm{e}$
  must have a real frequency at least $\theta$. This equals to say that all
  itemsets in $\mathcal{C}_\mathrm{e}$ are True Frequent Itemsets
  ($\mathcal{C}_\mathrm{e}\subseteq TFI(\theta)$.

  Given that $\mathsf{E}_\mathrm{v}$ occurs, then we know that all itemsets in
  $\mathcal{G}$ have frequency in $\Ds_\mathrm{v}$ that is at most
  $\varepsilon_\mathrm{v}$ far from their true frequency:
  \[
  \max_{A\in\mathcal{G}}\left|\tfreq(A)-f_{\Ds_\mathrm{v}}(A)\right|\le\varepsilon_\mathrm{v}\enspace.\]
  In particular this means that an  itemset $A\in\mathcal{G}$ can have
  $f_{Ds_\mathrm{v}}(A)\ge\theta+\varepsilon_\mathrm{v}$ \emph{only} if
  $\tfreq(A)\ge\theta$, that is \emph{only} if $A\in\TFI(\prob,\Itm,\theta)$.
  Hence, $\mathcal{C}_\mathrm{v}\subseteq TFI(\prob,\Itm,\theta)$.

  We can then conclude that if the event $\mathsf{E}$ occurs, we have
  $\mathcal{C}_\mathrm{e}\cup \mathcal{C}_\mathrm{v}\subseteq TFI(\prob,\Itm,\theta)$.
  Since $\mathsf{E}$ occurs with probability at least $1-\delta$, this equals to
  say that
  \[
  \Pr(\exists A\in \mathcal{T}_\mathrm{e}\cup
  \mathcal{T}_\mathrm{v} ~:~ \tfreq(A)<\theta)\le\delta\enspace.\]
\end{proof}

{\bf MATTEO:} Pseudocodice, menzione dell'algoritmo in giro per il paper,
risultati sperimentali, etc.
