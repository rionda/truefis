\subsection{An holdout approach}\label{sec:holdout}
We now present and analyze a second method, called \ALGHOLDOUT{}, for
mining the TFIs. \ALGHOLDOUT{} draws inspiration from the holdout technique
presented by~\citet{Webb07}, and requires that the dataset $\Ds$ can be
split into two parts (not necessarily of the same size) that can be seen as two
independent collections of i.i.d.~samples from the generating distribution
$\prob$: an \emph{exploratory} part $\Ds_\mathrm{e}$ and an \emph{evaluation}
part $\Ds_\mathrm{v}$.  \ALGHOLDOUT{} works in two phases: we first use the
exploratory part $\Ds_\mathrm{e}$ to identify a small set $\mathcal{G}$ of
candidate TFIs, and then we decide which of the candidates to include in the
output using their frequencies in the evaluation part $\Ds_\mathrm{v}$.

\paragraph{First phase.} Let $\delta_\mathrm{e}$ and $\delta_\mathrm{v}$ be such
that $(1-\delta_\mathrm{e})(1-\delta_\mathrm{v})\ge(1-\delta)$. Let
$\range(2^\Itm)$ be the range space of all itemsets. We can compute an upper
bound $d_\mathrm{e}$ to $\EVC(\range(2^\Itm), \Ds_\mathrm{e})$ as discussed in
Sect.~\ref{sec:computvcexact}. Then, using~\eqref{eq:evceapproxitemsets} we can compute an
$\varepsilon_\mathrm{e}$ such that $\Ds_\mathrm{e}$ is, with probability at
least $1-\delta_\mathrm{e}$, an $\varepsilon_\mathrm{e}'$-approximation to
$(\range(2^\Itm),\prob)$. These steps are exactly the same as the first steps of
\ALG{}, which operated on the \emph{whole} dataset, not just on one part like
\ALGHOLDOUT{}. We then compute the
collections of itemsets
\[
	\mathcal{C}_\mathrm{e}=\{X\subseteq\Itm ~:~
	f_{\Ds_\mathrm{e}}(X)\ge\theta+\varepsilon_\mathrm{e}\}\]
and
\[
\mathcal{G}=\{X\subseteq\Itm ~:~ \theta\le
	f_{\Ds_\mathrm{e}}(X)<\theta+\varepsilon_\mathrm{e}\}\enspace.
\]
To obtain these sets, we extract the set $\FI(\Ds_\mathrm{e},\Itm,\theta)$ and
partition it appropriately into $\mathcal{C}_\mathrm{e}$ and $\mathcal{G}$.

\paragraph{Second phase.} In the second phase, we compute, using one of the
methods from Sect.~\ref{sec:computvcexact}, an upper bound $d_\mathrm{v}$ to
$\EVC(\range(\mathcal{G}),\Ds_\mathrm{v})$, the empirical VC-dimension of the
range space associated to $\mathcal{G}$ on the evaluation part $\Ds_\mathrm{v}$.
Using this quantity in~\eqref{eq:evceapproxitemsets}, we then compute a value
$\varepsilon_\mathrm{v}$ such that $\Ds_\mathrm{v}$ is, with probability at
least $1-\delta_\mathrm{v}$, an $\varepsilon_\mathrm{e}$ approximation to
$(\range(\mathcal{G},\prob)$. We then compute the set
\[
\mathcal{C}_\mathrm{v}=\{X\subseteq\Itm ~:~ X\in\mathcal{G} \mbox{ and }
f_{\Ds_\mathrm{v}}(X)\ge\theta+\varepsilon''\}=\mathcal{G}\cap\FI(\Ds_\mathrm{v},\Itm,\theta+\varepsilon_\mathrm{v})\enspace.
\]
\ALGHOLDOUT{} returns the collection of itemsets $\mathcal{C}_\mathrm{e}\cup\mathcal{C}_\mathrm{v}$.

The pseudocode for the method is presented in Alg.~\ref{alg:holdout}, while the
following theorem shows the correctness of \ALGHOLDOUT.

\begin{theorem}
	With probability at least $1-\delta$,
	$\mathcal{C}_\mathrm{e}\cup\mathcal{C}_\mathrm{v}$ contains no false
	positives:
	\[
		\Pr(\exists A\in\mathcal{C}_\mathrm{e}\cup\mathcal{C}_\mathrm{v} \mbox{
		s.t. } \tfreq(A)<\theta)\le\delta\enspace.
	\]
\end{theorem}
\begin{proof}
  Consider the two events $\mathsf{E}_\mathrm{e}$=``$\Ds_\mathrm{e}$ is an
  $\varepsilon_\mathrm{e}$-approximation for $(\range(2^\Itm),\prob)$'' and
  $\mathsf{E}_\mathrm{v}=$``$\Ds_\mathrm{v}$ is an
  $\varepsilon_\mathrm{v}$-approximation for $(\range(\mathcal{G}),\prob)$. From
  the above discussion it follows that the event
  $\mathsf{E}=\mathsf{E}_\mathrm{e}\cap\mathsf{E}_\mathrm{v}$ occurs with probability at least
  $1-\delta$. Suppose from now on that indeed $\mathsf{E}$ occurs.

  Given that $\mathsf{E}_\mathrm{e}$ occurs, then all the itemsets with
  frequency in $\Ds_\mathrm{e}$ at least $\theta+\varepsilon_\mathrm{e}$
  must have a real frequency at least $\theta$. This equals to say that all
  itemsets in $\mathcal{C}_\mathrm{e}$ are True Frequent Itemsets
  (i.e., $\mathcal{C}_\mathrm{e}\subseteq\TFI(\theta)$).

  Given that $\mathsf{E}_\mathrm{v}$ occurs, then we know that all itemsets in
  $\mathcal{G}$ have frequency in $\Ds_\mathrm{v}$ that is at most
  $\varepsilon_\mathrm{v}$ far from their true frequency:
  \[
  \max_{A\in\mathcal{G}}\left|\tfreq(A)-f_{\Ds_\mathrm{v}}(A)\right|\le\varepsilon_\mathrm{v}\enspace.
  \]
  In particular this means that an itemset $A\in\mathcal{G}$ can have
  $f_{\Ds_\mathrm{v}}(A)\ge\theta+\varepsilon_\mathrm{v}$ \emph{only} if
  $\tfreq(A)\ge\theta$, that is, \emph{only} if $A\in\TFI(\prob,\Itm,\theta)$.
  Hence, $\mathcal{C}_\mathrm{v}\subseteq\TFI(\prob,\Itm,\theta)$.

  We can then conclude that if the event $\mathsf{E}$ occurs, we have
  $\mathcal{C}_\mathrm{e}\cup \mathcal{C}_\mathrm{v}\subseteq TFI(\prob,\Itm,\theta)$.
  Since $\mathsf{E}$ occurs with probability at least $1-\delta$, this equals to
  say that
  \[
  \Pr(\exists A\in \mathcal{C}_\mathrm{e}\cup
  \mathcal{C}_\mathrm{v} ~:~ \tfreq(A)<\theta)\le\delta\enspace.\]
\end{proof}


\begin{algorithm}[htbp]
  \SetKwInOut{Input}{Input}
  \SetKwInOut{Output}{Output}
  \SetKwComment{Comment}{\quad// }{}
  \SetKwFunction{SolveAntichainSUKP}{solveAntichainSUKP}
   \DontPrintSemicolon
   \Input{Dataset $\Ds$, freq.~threshold $\theta\in(0,1)$, confidence
   $\delta\in(0,1)$}
  \Output{Freq.~threshold $\hat{\theta}$
  s.~t.~$\FI(\Ds,\Itm,\hat{\theta})$ contains only TFIs with prob.~at least
  $1-\delta$.}
  $\Ds_\mathrm{e},\Ds_\mathrm{v}\leftarrow$ partitioning of $\Ds$ into two independent parts.\;
  $\delta_\mathrm{e},\delta_\mathrm{v}\leftarrow 1-\sqrt{1-\delta}$ \Comment{$\delta_\mathrm{e}$ and $\delta_\mathrm{v}$ do not need to have the same value}
  $d_\mathrm{e}\leftarrow$ upper bound to $\EVC(\range(2^\Itm),\Ds_\mathrm{e})$  \Comment{as
	discussed in Sect.~\ref{sec:computvcexact}}
	$\varepsilon_\mathrm{e}\leftarrow
	2c\sqrt{\frac{2d_\mathrm{e}\max_{a\in\Itm}f_{\Ds_\mathrm{e}}(\{a\})}{|\Ds_\mathrm{e}|}}
	+ \sqrt{\frac{2\ln\frac{4}{\delta_\mathrm{e}}}{|\Ds_\mathrm{e}|}}$\;
  $\mathcal{C}_\mathrm{e}=\FI(\Ds_\mathrm{e},\Itm,\theta+\varepsilon_\mathrm{e})$\;
  $\mathcal{G}=\{A\subseteq\Itm ~:~ \theta\le f_\Ds(A)<\theta+\varepsilon_\mathrm{e}\}$\;
  $d_\mathrm{v}\leftarrow$ upper bound to $\EVC(\range(\mathcal{G}),\Ds_\mathrm{v})$  \Comment{as
	discussed in Sect.~\ref{sec:computvcexact}}
	$\varepsilon_\mathrm{v}\leftarrow
	2c\sqrt{\frac{2d_\mathrm{v}\max_{A\in\mathcal{G}}f_{\Ds_\mathrm{v}}(A)}{|\Ds_\mathrm{v}|}}
	+ \sqrt{\frac{2\ln\frac{4}{\delta_\mathrm{v}}}{|\Ds_\mathrm{v}|}}$\;
  $\mathcal{C}_\mathrm{v}=\FI(\Ds_\mathrm{e},\Itm,\theta+\varepsilon_\mathrm{v})$\;
  \Return{$\mathcal{C}_\mathrm{e}\cup\mathcal{C}_\mathrm{v}$}
  \caption{BLABLA}
  \label{alg:holdout}
\end{algorithm}

{\bf MATTEO:} Menzione dell'algoritmo in giro per il paper, risultati
sperimentali, etc. Dobbiamo anche unificare la presentazione dei due algoritmi
perche' questo ritorna una collezione di TFIs, mentre quello prima ritorna un
frequency threshold. Inoltre, se reso necessario dagli esperimenti, possiamo
menzionare che per calcolare $\varepsilon_\mathrm{v}$ uno puo' usare un binomial
test.
